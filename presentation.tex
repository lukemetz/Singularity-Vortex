\documentclass{beamer}
\usetheme{default}
\begin{document}

\begin{frame}{}

\title{\textbf{Image Processing}}
\maketitle

Chloe Eghtebas, Luke Metz, Brendan Ritter

\end{frame}


\begin{frame}{Basics of Image Processing}


Using computational software to manipulate an image using signal processing techniques.

Discuss importance of image processing in modern society. Research it's applications. 

\end{frame}


\begin{frame}{What we Have Done}


\begin{itemize}
	\item[1]Image Compression
	\item SVD
	\item[2]"Signal Processing" 
	%\item Convolution
	%\item Gaussian blur
	\item Shear
    \item brightness and contrast 
    \item grayscale
    \item invert
    \item Rotating
    	\item Flipping
    \item[3]Edge Detection 
    \item Gaussian blur
    \item Sobel/convolution

\end{itemize}
\end{frame}

\begin{frame}{Image Representation and Software}
Images have 3 channels. Red, Blue, Green. To manipulate these arrays we chose to not use matlab. We chose to use Python and Numpy mainly for its re usability.
\end{frame} 


\begin{frame}{Flipping}

\end{frame}


\begin{frame}{Transformations}

\end{frame}

\begin{frame}{Convolution}
\end{frame}

\begin{frame}{Gaussian Blur}

\end{frame}

\begin{frame}{Sobel Edge Detection}

\end {frame}

\begin{frame}{Blur + Edge Detection}

\end{frame}


\begin{frame}{Color}

\end{frame}

\end{document}
