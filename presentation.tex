\documentclass{beamer}
\usetheme{default}

\usepackage{subfigure}
\usepackage{amsmath}
\usepackage{gensymb}

\begin{document}

\begin{frame}{}

\title{\textbf{Image Processing}}
\maketitle

Chloe Eghtebas, Luke Metz, Brendan Ritter

\end{frame}


\begin{frame}{Basics of Image Processing}

Using computational software to manipulate an image using signal processing techniques. 

It's applications: 
Biomedical Imaging
 Recognition and Tracking

\end{frame}



\begin{frame}{What You'll Need to Know}

\begin{itemize}
		
	\item[1]Image Compression
	\item SVD
	\item[2]"Signal Processing" 
	\item Shear
    \item brightness and contrast 
    \item grayscale
    \item invert
    \item Rotating
    	\item Flipping
    \item[3]Edge Detection 
    \item Gaussian blur
    \item Sobel/convolution
\end{itemize}
\end{frame}

\begin{frame}{Image Representation and Software}

\begin{figure}
\begin{center}
\includegraphics[width = 0.75 in]{bunny.jpg}
\end{center}
\end{figure}

Images have 3 channels. Red, Blue, Green. To manipulate these arrays we chose to not use matlab. We chose to use Python and Numpy mainly for its re usability.

\begin{figure}
\includegraphics[width = 0.75 in]{bunnyred.png}
\hspace{0.5 in}
\includegraphics[width = 0.75 in]{bunnygreen.png}
\hspace{0.5 in}
\includegraphics[width = 0.75 in]{bunnyblue.png}
\end{figure}

\end{frame} 


\begin{frame}{Flipping}


\begin{figure}
\includegraphics[width = 1.25 in]{lennastory.jpg}

\hspace{0.5 in}
\includegraphics[width = 1.25 in]{lenna1.png}
\hspace{0.5 in}
\includegraphics[width = 1.25 in]{lenna2.png}
\hspace{0.5 in}
\end{figure}

\end{frame}


\begin{frame}{Transformations}
To apply more complicated transformations to the image one first has to re arrange the image. Currently x and y position are stored at the index of the matrix. This gives us no easy way to manipulate them.

We can rearrange The image to look like:
$$A = \begin{pmatrix}
	x & 0 & ...\\
	y & 0 & ...\\
	1 & 1 & ...\\
	r &  255 & ...\\
	g & 255 & ...\\
	b & 255 & ...\\
\end{pmatrix}$$
This will let us translate, rotate, scale, and shear the image in any way.

\end{frame}

\begin{frame}{Rotation}
Example Rotation:
The Default rotation is always around the origin. To make that the center of the image, one must first translate, rotate, then translate back.
\hspace{0.1 in}
\\
$A_{out} = \begin{pmatrix}
	1 & 0 & dx\\
	0 & 1 & dy\\
	0 & 0 & 1\\
\end{pmatrix}$
$\begin{pmatrix}
	\cos(\theta) & -\sin(\theta) & 0\\
	\sin(\theta)& \cos(\theta) & 0\\
	0 & 0 & 1\\
\end{pmatrix}$
$\begin{pmatrix}
	1 & 0 & -dx\\
	0 & 1 & -dy\\
	0 & 0 & 1\\
\end{pmatrix}$
$A$
\\
\begin{figure}


\includegraphics[width = 1.1 in]{bunnycute.jpg}
\hspace{0.5 in}
\includegraphics[width = 1.1 in]{bunnycuteRot.jpg}
\caption{Rotated around the center at $\theta = 45\degree$}
\end{figure}
\end{frame}


\begin{frame}{Rotation}
Example Rotation:
The Default rotation is always around the origin. To make that the center of the image, one must first translate, rotate, then translate back.
\hspace{0.1 in}
\\
$A_{out} = \begin{pmatrix}
	1 & \lambda_y & 0\\
	\lambda_x & 1 & 0\\
	0 & 0 & 1\\
\end{pmatrix}$
$A$
\\
\begin{figure}

\includegraphics[width = 1.1 in]{bunnycute.jpg}
\hspace{0.5 in}
\includegraphics[width = 1.1 in]{bunnycuteShear.jpg}
\caption{Sheared with $\lambda_X = .4$ and $\lambda_Y = 0$}
\end{figure}

\end{frame}




\begin{frame}{Convolution}
'Blending' two functions together. For discrete data, its like overlapping.
One has original data and a kernel which represents the other function. 
\begin{figure}[htp]
\centering
\includegraphics[width=2in]{conv2d_matrix.jpg}
\caption{Convolution in 2D.}
\label{}
\end{figure}
\end{frame}

\begin{frame}{Gaussian Blur}

Gaussian blur is an application of  convolution. It 'blends' a Gaussian, a normal curve onto the image. One has control over blurriness by the standard deviation, $\sigma$.

%\begin{figure}[ht]
%\begin{minipage}[b]{0.1\linewidth}
%\centering
%\includegraphics[width=1.3in]{churchin.jpg}
%\caption{Default Image}
%\label{fig:figure1}
%\end{minipage}
%\hspace{2.0in}
%\begin{minipage}[b]{0.1\linewidth}
%\centering
%\includegraphics[width=1.3in]{churchoutb.jpg}
%\caption{Blurred Image}
%\hspace{1.0in}
%\label{fig:figure2}
%\end{minipage}
%\end{figure}

\begin{figure}[ht]
\includegraphics[width=1.3in]{churchin.jpg}
\hspace{.1in}
\includegraphics[width=1.3in]{churchoutblur.jpg}
\hspace{.1in}
\includegraphics[width=1.3in]{churchoutblur2.jpg}
\caption{Original image on left. $\sigma$ = 2 in middle. $\sigma$ = 4 on left.}
\end{figure}


\end{frame}

\begin{frame}{Sobel Edge Detection}
Another application of convolution is edge detection. The Sobel operator can be thought of as a discrete differentiation operator. It gets how fast one color goes to another. 

\begin{figure}[ht]
\includegraphics[width=1.4in]{churchin.jpg}
\hspace{.1in}
\includegraphics[width=1.4in]{churchout.jpg}
\hspace{.1in}
\end{figure}
\end {frame}

\begin{frame}{Sobel Kernel}
The kernel used to create this edge detection is in two parts:


$K_X = \begin{pmatrix}
	-1 & 0 & 1\\
	-2 & 0 & 2\\
	-1 & 0 & 1\\
\end{pmatrix}$
$K_Y = \begin{pmatrix}
	-1 & -2 & -1\\
	0 & 0 & 0\\
	1 & 2 & 1\\
\end{pmatrix}$

\vspace{.4in}
$E = \sqrt(E_Y^2+E_Y^2)$ Where $E_X$ and $E_Y$ are the result of the convolution.

\end{frame}

\begin{frame}{Blur + Edge Detection}
One can combine these operations to find the major edges in an image.

\begin{figure}[ht]
\includegraphics[width=1.3in]{churchout.jpg}
\hspace{.1in} 
\includegraphics[width=1.3in]{churchoutbluredge.jpg}
\hspace{.1in}
\includegraphics[width=1.3in]{churchoutblur2edge.jpg}
\hspace{.1in}
\caption{No Blur, Blur, $\sigma = 2$, $\sigma = 4$}
\end{figure}

\end{frame}


\begin{frame}{Color}

\end{frame}

\end{document}
